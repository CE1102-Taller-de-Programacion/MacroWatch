\documentclass{article}[12pt]
    \usepackage[utf8]{inputenc}
    \title{Macro Watch}
    \author{David Azofeifa}
    \date{II Semetre, 2017}
    \begin{document}
    \maketitle
    \tableofcontents
    \section{Introducción}
    \p
    El proyecto "Macro Watch" corresponde a la tarea programada I, del curso: CE-1102, Taller de Programación. Este se plantea la simulación de una interfaz de un reloj inteligente. La interfaz permite al usuario a accesar a sus datos y accionar operaciones del sistema para manipularlos, entre otras acciones. La interfaz no requier conocimiento técnico de parte del usuario, permitiendo una gran accesibilidad. La totalidad de la aplicación sería conformada por la interfaz, las operaciones por debejo de ella y los archivos necesarios para guardar los metadatos. Por lo tanto, la aplicación permite al usuario acceder, modificar y eliminar su información sin tener que precuparse de que una vez la memoria principal sea borrada, estos desaparezcan.

    \section{Descripción del problema}
    El programador se plantea la incógnita obtener las herramientas y conocimiento necesario para crear la aplicación. ¿Cuales bibliotecas permiten la manipulación de datos en disco? ¿Como se implementa de una forma cómoda y flexible otro idioma dentro de la interfaz?
    
    \section{Análisis de resultados}

    \section{Bitácora de actividades}

    \subsection{Estadística de tiempos}

    \section{Conclusiones}
\end{document}